\thechapter{Introduction} 
In the field of computer graphics, non-photorealistic rendering (NPR) is a somewhat circular term that 
refers to the use of rendering techniques to achieve a non-photorealistic effect. There are several 
situations in which a non-photorealistic rendering is advantageous. 

The first field where this is commonly applied is technical illustration, as described by Gooch et al. 
in \cite{gooch98}. Technical illustrations are commonly found
in manuals for products that need to be disassembled for repairs, or in training documentation for a 
particular device. As such, the depictions of the device are static images, and it is often the goal of
the illustrators to convey as much information as possible with these static images. Often, realistic
shading is omitted, and image are drawn with an isometric perspective, so the scale and shape of 
geometry is easier to read. In addition, outlines are commonly added. If a light source is used, 
sometimes a technique called "rim lighting" is applied that highlights edges of faces that is pointed 
away from a light source.

Another field in which NPR is commonly used is in interactive media. Games and animation that use NPR
often apply it as an artistic decision. A realistic rendering may be seen as
too visually busy, or does not provide the user with as much visual information as the designers wish.
It may also be that designers wish to imitate a style closer to that of cartoons, for one reason or 
another. Cartoons are generally drawn in a wide variety of styles, but most have the following 
characteristics: Foreground objects such as characters are drawn with outlines. Most foreground objects
are drawn with a single colour, and shading is shown with a darker version of this colour. This 
technique is referred to as cel shading, and is very common in many stylized modern games, including
examples such as Sable.

The goal behind this project is to write a program that applies common NPR effects to 3D models, to 
create our own NPR rendering of the models. In particular, we will implement cel shading, object 
contours, and rim highlights to achieve our final NPR rendering of our models. 
