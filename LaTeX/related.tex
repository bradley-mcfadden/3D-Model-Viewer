\thechapter{Prior and related work}
Gooch et al.'s work in technical illustration demonstrates many of the common 
non-photorealistic rendering effects. The authors made the observation that there was a 
disconnect between the style of artists and illustrators, and the style created by computer 
graphics programmers. Where computer graphics artists were concerned with techniques to 
create a realistic image, illustrators of technical manuals opt fir a style that emphasizes
the geometry of an object, while removing extraenous detail. The authors note that technical
illustration needs this characteristic, as the viewer does not have the ability to move
around a printed image to get more information. To produce a shaded technical rendering of
an object, Gooch et al. render objects with black edge lines, no shadowing, intensities far 
from black or white, colour indicated by surface normal, and a single light source that 
provides white highlights (also known as rim lighting). It is interesting to note that the 
authors provide a method for approximating their model using phong shading with negative 
light colours.

The work of DeCarlo et al. starts from a similar question \cite{decarlo03}, how can the shape 
of a 3D object be conveyed with just lines? The research describes two techniques to display 
an object's shape, namely contours and suggestive contours. Contours describe an area of an 
image where a surface turns away from the viewer, becoming invisible. Similarly, suggestive
contours are like contours, but describe areas where the surface of an object turns away from
viewer, yet remains visible. Additionally, Decarlo et al. provide mathematical definitions
and algorithms that detail how contours and suggestive contours can be produced on a model.

Decaudin et al. draw upon several techniques to render 3D scenes in a cartoon style 
\cite{decaudin96}. Their work presents a rendering algorithm that proceeds in four stages. 
Firstly, they render the scene with ambient lighting. Next, they find the outlines of each 
object in the scene. For each light source, they render the scene as illuminated by it, and 
then find the project shadows from other objects. Finally, they combine the steps to produce 
a cartoon style image. Notably, their ambient lighting stage renders each object with a 
single uniform colour, and the shadows of the scene are also a single colour, which seems 
like a form of cel shading.