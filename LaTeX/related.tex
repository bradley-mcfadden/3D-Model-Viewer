\thechapter{Prior and related work}
Gooch et al.'s work in technical illustration demonstrates many common non-photorealistic 
rendering effects \cite{gooch98}. The authors made the observation that there was a 
disconnect between the style of artists and illustrators, and the style created by computer 
graphics programmers. Where computer graphics artists were concerned with techniques to 
create a realistic image, illustrators of technical manuals opt for a style that emphasizes
the geometry of an object, while removing extraneous detail. The authors note that technical
illustration needs this characteristic, as the viewer does not have the ability to move
around a printed image to get more information. To produce a shaded technical rendering of
an object, Gooch et al. render objects with black edge lines, no shadowing, intensities far 
from black or white, colour indicated by surface normal, and a single light source that 
provides white highlights (also known as rim lighting). It is interesting to note that the 
authors provide a method for approximating their model using Phong shading with negative 
light colours.

The work of Gooch is very similar to \cite{saito90}. Saito and Takahashi also sought to create
more clearly comprehensible renderings of 3D shapes, in particular renderings that maintained
their clarity after multiples passes through a copy machine. Their work focuses on image 
processing techniques that occur after an existing rendering has been produced. The authors 
combine image enhancement techniques together to produce more clear images. They find edges, 
contours, discontinuities, and curved hatching together to produce images of 3D objects with
enhanced edges, and more clear shape from curved hatching. Additionally, they demonstrate
their results on topographical data, to produce easy to read topographical maps of terrain.

The work of DeCarlo et al. starts from a similar question, how can the shape of a 3D object be
conveyed with just lines \cite{decarlo03}? The research describes two techniques to display 
an object's shape, namely contours and suggestive contours. Contours describe an area of an 
image where a surface turns away from the viewer, becoming invisible. Similarly, suggestive
contours are like contours, but describe areas where the surface of an object turns away from
viewer, yet remains visible. Additionally, Decarlo et al. provide mathematical definitions
and algorithms that detail how contours and suggestive contours can be produced on a model.

Decaudin et al. draw upon several techniques to render 3D scenes in a cartoon style 
\cite{decaudin96}. Their work presents a rendering algorithm that proceeds in four stages. 
Firstly, they render the scene with ambient lighting. Next, they find the outlines of each 
object in the scene. For each light source, they render the scene as illuminated by it, and 
then find the project shadows from other objects. Finally, they combine the steps to produce 
a cartoon style image. Notably, their ambient lighting stage renders each object with a 
single uniform colour, and the shadows of the scene are also a single colour, which seems 
like a form of cel shading.

Work by Mitchell et al. describes how a modified Phong shading model can be used to increase visual
clarity of rendered models in a first-person actions game like Team Fortress 2 \cite{mitchell07}. One 
technique employed is a transformation on the dot product of vector $n$ and $I$ in 
the diffuse lighting equation. The transformation prevents models from losing shape information
on faces opposite a particular light source. The authors transform the Phong model
to add a function that takes the scale produced by diffuse lighting, and breaks it into three 
regions, a dark gray end ground, a light gray start zone, and a middle ground with a slight 
red component. This is done from observations that artists tended to favour use light grays and
dark grays instead of black and white, and preferred to mix in warm colours to their mid tones. 
The authors also add a dedicated rim lighting term to their Phong model in order to make upward
facing surfaces more likely to be rim shaded.

To create images that are even closer to traditional hand drawn line renderings Al-Rousan et al. used 
Laplacian smoothing to reduce rough and sharp edges of models, creating a simplified model to reduce 
details in a realistic way before using other algorithms to outline contours of the model \cite{rousan16}. 
Similarly, Lee et al. focussed on improving directional lighting through dividing objects into curved 
surface patches, then applying lighting effects to each surface patch independently to optimally 
highlight details and draw the viewer's eye \cite{lee06}. In both cases simplification and modification 
of the initial model allowed for clearer renderings, but also required complicated pre-processing to the 
model.